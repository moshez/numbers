\usepackage{listings}
\usepackage{textcomp}
\usepackage{fancyvrb}

\title{Everything You Hoped You'll Never Have to Know About Numbers in Python}
\subtitle{...but you really do}
\author{Moshe Zadka -- https://cobordism.com}
\date{2020}

\begin{document}
\begin{titlepage}
\maketitle
\end{titlepage}

\frame{\titlepage}

\begin{frame}
\frametitle{Numbers in Math}
"God gave us the natural numbers;
the rest of it we have only ourselves to blame" --
with apologies to Leopold Kronecker

\begin{itemize}
\item Natural numbers
\item Integers
\item Rationals
\item Real numbers
\item Complex numbers
\item Are quaternions numbers? Who knows!
\end{itemize}
\end{frame}


\begin{frame}[fragile]
\frametitle{Numbers in Python}

\begin{itemize}
\item \texttt{int}
\item \texttt{float}
\item \texttt{complex}
\item \texttt{fractions}
\item \texttt{decimal}
\item Is \texttt{gmpy2} relevant? Who knows!
\end{itemize}

\end{frame}

\begin{frame}[fragile]
\frametitle{Integers, division, and multiplication}

\begin{lstlisting}
>>> (4/3) * 3
4.0
\end{lstlisting}
\end{frame}

\begin{frame}[fragile]
\frametitle{floats}

Just pay a \$100 for IEEE-754...\pause

(if you did it before 2019, pay again, they revised it)...\pause

read all 70 pages...\pause

and memorize them.
\end{frame}

\begin{frame}[fragile]
\frametitle{floats: IEEE-754}

\begin{itemize}
\item roundTiesToEven, the floating-point number nearest to the
infinitely precise result shall be delivered;
if the two nearest floating-point numbers bracketing an unrepresentable
infinitely precise result are equally near,
the one with an even least significant digit shall be delivered
\end{itemize}

\end{frame}

\begin{frame}[fragile]
\frametitle{floats: surprise!}

\begin{lstlisting}
>>> 1 + 2 - 2 - 1
0
>>> 0.1 + 0.2 - 0.2 - 0.1
2.7755575615628914e-17
\end{lstlisting}

\end{frame}

\begin{frame}[fragile]
\frametitle{floats: surprise strikes again!}

\begin{lstlisting}
>>> a = 2**-53
>>> (a + a) + 1 == a + (a + 1)
False
\end{lstlisting}
\end{frame}

Adding 10K numbers took over a minute on my gaming laptop.

\begin{frame}[fragile]
\frametitle{fractions: a different sort of surprise}

\begin{lstlisting}
>>> before = datetime.now()
>>> res = sum(inverses[:10000])
>>> after = datetime.now()
>>> print("It took", after-before)
>>> print("Size of output", len(str(res)))
>>> print("Approximate value", float(res))
It took 0:01:08.524281
Size of output 90743
Approximate value 1.2538568497816165
\end{lstlisting}
\end{frame}

Converting to floats before adding took so little time,
I don't trust the number.

\begin{frame}[fragile]
\frametitle{fractions: a different sort of surprise}

\begin{lstlisting}
>>> before = datetime.now()
>>> res = sum(map(float, inverses[:10000]))
>>> after = datetime.now()
>>> print("It took", after-before)
>>> print("Size of output", len(str(res)))
>>> print("Approximate value", float(res))
It took 0:00:00.003096
Size of output 18
Approximate value 1.2538568497816087
\end{lstlisting}
\end{frame}

\begin{frame}[fragile]
\frametitle{fractions: a different sort of surprise}

\begin{lstlisting}
Approximate value 1.2538568497816165 # With precise fractions
Approximate value 1.2538568497816087 # With floats
                    12345678901234
\end{lstlisting}
\end{frame}

\begin{frame}[fragile]
\frametitle{Decimals: Hidden state}

Quote from documentation:

\begin{lstlisting}
>>> getcontext().prec = 6
>>> Decimal(1) / Decimal(7)
Decimal('0.142857')
>>> getcontext().prec = 28
>>> Decimal(1) / Decimal(7)
Decimal('0.1428571428571428571428571429')
\end{lstlisting}

\end{frame}

\begin{frame}[fragile]
\frametitle{Decimals: Hidden state}

Fixed it for you

\begin{lstlisting}
>>> getcontext().prec = 6
....121 lines...
>>> Decimal(1) / Decimal(7)
Decimal('0.142857')
>>> getcontext().prec = 28
....537 lines...
>>> Decimal(1) / Decimal(7)
Decimal('0.1428571428571428571428571429')
\end{lstlisting}

\end{frame}

\begin{frame}[fragile]
\frametitle{Decimals: Using correctly}

\begin{lstlisting}
>>> getcontext().prec = 6
>>> # 6853 lines elided
... with localcontext() as ctx:
...     ctx.prec = 10
...     Decimal(1) / Decimal(7)
... 
Decimal('0.1428571429')
\end{lstlisting}

\end{frame}

\end{document}
